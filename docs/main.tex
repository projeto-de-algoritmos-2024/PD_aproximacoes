\documentclass{article}
\usepackage[utf8]{inputenc}
\usepackage{graphicx} % Required for inserting images
\usepackage{amsmath}
\usepackage[brazil]{babel}

\title{RELATÓRIO II: COMPARAÇÃO DE PERFORMANCE ÓTIMO VS APROXIMADO PARA PD}
\author{
    Ryan Augusto Brandão Salles - 221008436\\
    Víctor Moreira Almeida - 221008481\\
}

\date{Fevereiro de 2024}

\begin{document}

\maketitle

\section{Introdução}
Essa seção visa trazer alguns assuntos introdutórios, mais precisamente, situar o leitor sobre o que se trata esse documento, qual o trabalho que busca ser realizado e qual exatamente seria sua motivação, além de demais especificidades de caráter estritamente introdutório ao teor do que será tratado nesse relatório.

\subsection{O que é esse documento}
Esse documento é um relatório que sumariza os resultados obtidos durante a implementação e testagem de algoritmos de otimização para o problema da Mochila (Knapsack) na disciplina de Projeto de Algoritmos durante o semestre 2024.2. Além disso, são detalhadas as abordagens utilizadas na implementação desses algoritmos.

Mais especificamente, o relatório apresenta uma comparação entre algoritmos exatos e heurísticos, com ênfase na otimização do problema da mochila utilizando Programação Dinâmica e sua comparação com uma abordagem gulosa (Greedy). Detalhes sobre esses algoritmos podem ser encontrados na seção \ref{algoritmos}.

Ademais, os algoritmos foram escolhidos para explorar o equilíbrio entre exatidão e eficiência computacional, com foco na análise de desempenho e qualidade das soluções obtidas, sem necessariamente visar um código pronto para produção.

\subsection{O que não é esse documento}
Esse documento não tem como objetivo ser uma pesquisa formal sobre a otimização do problema da Mochila (Knapsack) nem apresentar uma implementação completa e definitiva. Trata-se apenas de um experimento exploratório realizado pela dupla, cujo intuito é compartilhar observações e resultados que possam ser de interesse para seus colegas e avaliadores.

Além disso, este relatório não pretende ser uma prova de que a Programação Dinâmica é sempre superior a abordagens heurísticas, como o algoritmo guloso (Greedy). A eficiência de cada método depende do contexto e das restrições do problema, sendo necessário um estudo mais aprofundado para determinar a melhor abordagem em cada caso específico.

\subsection{Justificativa I: possibilidade de melhor desempenho e suas consequências}
O desempenho de algoritmos de otimização pode ser crucial para problemas de grande escala. Melhor eficiência pode significar a diferença entre obter uma solução em poucas horas ou precisar de anos de processamento. No caso do problema da Mochila (Knapsack), técnicas como Programação Dinâmica podem oferecer ganhos significativos em relação a abordagens heurísticas, tornando possível resolver instâncias maiores de forma mais viável.

\subsection{Justificativa II: interesse acadêmico e lúdico}
Algoritmos acadêmicos são tradicionalmente implementados de maneira sequencial, explorando apenas abordagens clássicas como Programação Dinâmica ou Heurísticas Gulosas. Nosso objetivo foi investigar as diferenças de desempenho entre essas abordagens e compreender melhor seus impactos práticos.

Mais especificamente, o algoritmo guloso, apesar de eficiente, nem sempre encontra a solução ótima, enquanto a Programação Dinâmica pode garantir essa optimalidade, ainda que com maior custo computacional. Isso levanta a questão de quão grande é essa diferença na prática e até que ponto vale a pena sacrificar precisão por tempo de execução.

Além disso, simplesmente parece uma análise interessante e divertida de se realizar, o que por si só já seria uma justificativa válida—caso as \textbf{outras justificativas} não fossem suficientes.


\subsection{Escolha de linguagem}
A linguagem escolhida para essa implementação foi (novamente) a Python, por duas razões simples:
\begin{itemize}
    \item simplicidade de uso
    \item disponibilidade de utensílios
    \item facilidade de desenvolvimento rápido
\end{itemize}
Onde por "simplicidade de uso" queremos dizer "Nós não precisamos pensar muito em como algo precisa ser feito ou pesquisar sintaxe" e, por "disponibilidade de utensílios" queremos dizer "não é necessário escrever algo tão básico quanto uma lista encadeada ou um método pop, a linguagem já possui isso implementada na biblioteca padrão". O desenvolvimento rápido é meramente uma consequência das outras justificativas\\

\section{Metodologia}
Para esse experimento, a metodologia utilizada é o seguinte procedimento:
\begin{enumerate}
    \item implementação dos algoritmos do Knapsack
    \item avaliação da performance
    \item implementação dos algoritmos ótimo e aproximado por greed
    \item avaliação da performance dos algoritmos
    \item comparação de performances e determinação da acurácia do Greedy
\end{enumerate}
Esses itens servirão de subtópicos para a metodologia e serão expandidos a seguir.\\
Demais tópicos além dos enumerados serão adicionados conforme necessidade.\\
\\
Antes de iniciar a implementação, é importante definir o hardware que será utilizado para os cálculos, afim de evitar duvidas caso haja discrepância entre os resultaods.\\
Serão utilizados dois computadores:
\begin{itemize}
    \item Dell G15 5520.
\begin{itemize}
    \item Windows 10 Pro 22H2.
\end{itemize}
    \item SteamDeck LCD.
\begin{itemize}
    \item SteamOS Holo 3.6.20.
\end{itemize}
\end{itemize}

\subsection{Implementação dos algoritmos do Knapsack}
    A implementação do algoritmo base consiste em implementar o algoritmo em python inicialmente de forma ótima para idealmente obter um ótimo para comparação posterior com o algoritmo de aproximação. 

\subsection{Avaliação da performance base}
A avaliação da performance base se dará por rodar o algoritmo obtido na implementação base e observar como a curva de crescimento do tempo de execução se comporta para inputs cada vez maiores. Isso serve 2 propósitos simples:
\begin{itemize}
    \item Averiguar se a implementação feita foi o mais próxima possível de uma complexidade ótima (cuja negação implica na necessidade de corrigir o algoritmo para o caso ótimo);
\end{itemize}
e, certamente,
\begin{itemize}
    \item Obter a performance que desejamos melhorar com o uso de estratégias de aproximação, mais especificamente, buscamos diminuir o custo computacional à um custo de valor na bolsa do knapsack que deve ser levado em consideração.
\end{itemize}

    Mais especificamente, buscamos descobrir quão próximos do ótimo os algoritmos de aproximação chegam e quanto tempo a menos eles podem custar, assunto do nosso próximo tópico

\subsection{Método de comparação ótimo vs. aprox.}
    Para cada rodagem em um determinado conjunto de dados
\subsection{Conjunto de dados para testagem}
Utilizaremos a biblioteca time para obter o tempo de execução do algoritmo e um vetor de pares peso-valor, denominado item, gerado durante a execução que utilizará a biblioteca random. Os testes serão os seguintes:
    \begin{enumerate}
        \item 10 itens com 10 de peso máximo na mochila gerados com seed 10
        \item 100 itens com 100 de peso máximo na mochila gerados com seed 100
        \item 1000 itens com 1000 de peso máximo na mochila gerados com seed 1000
    \end{enumerate}
E assim por diante, aumentando progressivamente o conjunto de dados e a seed por um fator de 10. Caso o algoritmo demore mais que 30 segundos para executar, ele será considerado muito lento e sua execução será terminada. Portanto, 30 segundos de espera é a condição para time limit exceeded (TLE).

\section{Algoritmos}\label{algoritmos}
Nesse tópico, serão brevemente explicados os algoritmos implementados, seu funcionamento, objetivo e estratégia de resolução do problema.

\subsection{Knapsack - Versão dinâmica iterativa}
O algoritmo dinâmico iterativo para o knapsack busca prencher uma matriz $I\ X\ P$, onde I são os itens a serem considerados e P é o peso. Para cada célula da matriz, o algoritmo leva em consideração as seguintes questões sobre o item e o peso:
\begin{itemize}
    \item Considerando o peso remanescente, é possível colocar esse item na mochila?
    \item Considerando o que foi previamente calculado, vale a pena levar esse item em detrimento de outros itens?
\end{itemize}
Respondidas essas perguntas, o algoritmo decide entre levar ou não o item. Como a nossa implementação é a iterativa, sempre são respondidas essas perguntas para versões mais simples do problema em questão e o algoritmo deve aos poucos construir a versão mais completa do problema.\\
Isso implica em preencher cada célula da matriz, sempre do canto superior esquerdo para o canto inferior direito, onde reside a resposta ótima.\\
Para determinar quais itens foram pegos, um algoritmo secundário, o finder, é rodado após a geração da matriz. Ao contrário do algoritmo principal, ele roda a partir da célula de resposta e sempre busca responder:
\begin{itemize}
    \item A célula atual é maior que a célula diretamente superior?
\end{itemize}
Se sim, o item da linha específica foi colocado na mochila. Se não, ou seja, a célula diretamente acíma possui o mesmo valor da célula atual, o item não foi pego.

\subsection{Knapsack - Aproximação por estratégia gananciosa/greedy}
Um algoritmo completamente diferente e certamente mais simples, ele utiliza a estratégia de sortir os itens por uma taxa valor sobre peso, que chamaremos de vsp. Então, basta coletar os itens com a maior taxa vsp até que:
\begin{itemize}
    \item não hajam mais itens ou
    \item a mochila esteja cheia
\end{itemize}
Sua performance é altamente dependente da velociade do algoritmo de ordenação utilizado para obter a ordem em qual os itens devem ser processados. 

\section{Resultados}
Os resultados obtidos aqui presentes foram obtidos utilizando uma cpu Intel(R) Core(TM) i5-12500H para o script knapsack.py, disponível na pasta src.\\
Para fins de comparação completa, 
Tempos acíma de 30 segundos serão considerados TLE. Os valores utilizados estão em segundos.\\
\subsection{Singlethread}
    
    \begin{center}
        \begin{tabular}{||c c c c||} 
         \hline
         100 & 1k & 10k & 100k \\ [0.5ex] 
         \hline\hline
         0.0002 & 0.0024 & 0.0807 & 7.449 \\
         \hline
        \end{tabular}
    \end{center}
Enquanto o tempo de rodagem para 100k não demorou para o usuário precisamente 7.449 segundos, isso pode ser explicado por trocas de contexto do sistema.

\subsection{Multithread I - 4 Threads}
    \begin{center}
        \begin{tabular}{||c c c c||} 
         \hline
         100 & 1k & 10k & 100k \\ [0.5ex] 
         \hline\hline
         0.0024 & 0.0043 & 0.0901 & 8.9766 \\
         \hline
        \end{tabular}
    \end{center}

Enquanto a velocidade é inicialmente comparável, pode-se observar que criar as threads para eventualmente iniciar a mesma rodagem do algoritmo mergeSort possui um custo que não exatamente se paga.\\

\section{Conclusão}

A implementação desenvolvida utilizando threads resultou em um desempenho reduzido no código, caracterizando um slowdown. Especula-se que a paralelização por GPU ou a reescrita do código com uma abordagem mais eficiente para operações com threads poderia proporcionar melhorias na execução do algoritmo. No entanto, os testes realizados até o momento indicam que a paralelização, na configuração atual, tem contribuído para a diminuição da performance.
% amongus sussy baka >:3 
% ඞ

\end{document}
